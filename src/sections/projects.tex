\sectionTitle{Current Research Projects}{}
\begin{projects}

\project
    {Conversation Analysis as an Annotation System for Emotion Detection}{Aug'22 - Present}
	{ \textit{Advisor: \href{https://linguistics.illinois.edu/directory/profile/girju}{Roxana Girju}}}
	{
	\begin{itemize}
	\setlength\itemsep{0.3em}
     \item Training language models to detect emotion using the sociolinguists Conversation Analysis approach as the main framework supporting data annotation.
     \item Annotating phone calls by linking well-linguistically researched conversational cues to possible underlying emotions based on the conversation analysis literature.   %{\small{\lbrack\textbf{{AdaptNLP@EACL'21}}\rbrack}}
     \item Focusing on annotating emotions exhibited in turn taking and silence between two interlocutors. {\small{\lbrack\textbf{{Working Paper}}\rbrack} }
     \end{itemize}
     }
     
\project
	{Feature-rich Open-vocabulary Interpretable Representations: An Application on Arabic
}{Aug'22 - Present}
	{
	     \textit{Advisor: \href{http://dowobeha.github.io/about/}{Lane Schwartz}}
	}
	{\begin{itemize}
	\setlength\itemsep{0.3em}
     \item This project uses  \href{https://aclanthology.org/2022.fieldmatters-1.8/}{Feature-rich Open-vocabulary Interpretable Representation (FOIR)} designed to model words from all of the world’s languages to model different Arabic dialects. 
     \item Even in the absence of a digitized corpus, using Tensor Product Representations proposed by Smolensky (1990), we can encode complex morphological features regardless of the morphological framework of the language.~ [\href{https://github.com/neural-polysynthetic-language-modelling/iiksiin}{\small{\websiteSymbol}}]  ~
     \item{\href{https://aclanthology.org/2022.fieldmatters-1.8/}{Schwartz et al.(2022)} explore the application of FOIR on several languages with different morphological systems including Maltese, a non-concatenative morphology language. As Maltese originally developed from Arabic, this project further explores the applications of this model on Arabic and its different dialects.}  {\small{\lbrack\textbf{{Working Paper}}\rbrack} }
     %\item Understand and track the impact of different language-based technological interventions that were carried out by \href{http://cgnetswara.org/}{CGNet Swara} in the Gond Community of Chattisgarh. \href{https://en.wikipedia.org/wiki/Gondi_language}{Gondi} is a language spoken by 3 million people however is a severely low-resourced language mainly attributed to non-existence of its own script. {\small{\lbrack\textbf{{LREC'20}}\rbrack}} ~ {\small{\lbrack\textbf{{LT4All}}\rbrack}}
     %\item Surveyed the challenges faced for deployment of language technologies to marginalized communities. {\small{\lbrack\textbf{{ICON'19}}\rbrack}}
     %\item Coverage/Mentions - {ACL'20}: \href{https://qz.com/1920191/internet-translation-access-creates-a-powerful-digital-divide/}{Quartz},  \href{http://newsletter.ruder.io/issues/reviewing-taking-stock-theme-papers-poisoning-and-stealing-models-multimodal-generation-240687}{NLP Newsletter}, \href{https://ruder.io/nlp-beyond-english/}{ruder.io/nlp-beyond-english}, \href{https://www.underratedml.com/episodes/episode-05-language-independence-and-material-properties}{Underrated ML Podcast}, \href{https://lacunafund.org/language/}{Lacuna Fund}, \href{https://sigtyp.github.io/sigtyp-newsletter-Apr-2020.html}{SIGTYP Newsletter}; {LREC'20}: \href{http://toi.in/HcX74b/a31g}{Times of India}, \href{https://www.hindustantimes.com/india-news/gonds-in-chhattisgarh-get-app-for-news-in-their-language/story-uQEDqDGBIPty7rMCNskTsK.html}{Hindustan Times}, \href{https://www.etvbharat.com/hindi/chhattisgarh/state/raipur/now-tribal-can-hear-news-and-stories-in-their-language/ct20190802191729141}{ETV}
     \end{itemize}}


     \project
	{Mental Representation of Time in Arabic and English Bilinguals}{Mar'21 - Present}
	{
	    \textit{Advisor:  \href{https://linguistics.illinois.edu/directory/profile/tionin}{Tania Ionin}}
	}
	{\begin{itemize}
	\setlength\itemsep{0.3em}
     \item Research shows that speakers of languages with a left-to-right writing direction mentally organize time from left to right, and speakers of languages with a right-to-left writing direction organize time from right to left \href{https://onlinelibrary.wiley.com/doi/full/10.1111/j.1551-6709.2010.01105.x}{(Fuhrman \& Boroditsky, 2010)}.
     \item{This project uses PC-IBEX to conduct experiments exploring the effect of writing direction on mental representations of time among Arabic and English bilinguals in comparison to their monolingual control groups.}
 
 ~ [\href{https://github.com/maimm2/PCIbexExperiments}{\small{\websiteSymbol}}] ~ {\small{\lbrack\textbf{{Qualifying Examination Paper - 2021 \& In Submission}}\rbrack}}
    %% \item Engaging with non-profits \href{https://translatorswithoutborders.org/}{Translators without Borders}, Pratham Books' \href{https://storyweaver.org.in/}{Story Weaver} and \href{http://cgnetswara.org/}{CGNet Swara} (covered by \href{https://www.livemint.com/mint-lounge/features/now-a-unique-machine-translation-tool-from-hindi-to-gondi-11597386377981.html}{LiveMint}) looking at possible solutions for deploying INMT for low resource languages. ~ {\small{\lbrack\textbf{{In Submission}}\rbrack}}
     %%\item Developing new interfaces to tailor to specific use-cases of translation such as document translation and web-page localization (using \href{github.com/microsoft/inmt-browser}{browser extension}) and offline translation (\href{https://github.com/microsoft/INMT-lite}{INMT lite}).
     \end{itemize}}
\end{projects}    
\vspace{-3mm}

